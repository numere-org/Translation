%
% A. DOKUMENTKLASSE
% ---------------------------------------------------------------------------
%

%
%  		Festlegen der Zeichencodierung des Dokuments und des Zeichensatzes.
%     Wir verwenden 'Latin1' (ISO-8859-1) f�r das Dokument,
%     und die 'T1' codierung f�r die Schrift.
%
\usepackage[latin1]{inputenc}
\usepackage[T1]{fontenc}
%
%  		Paket f�r die Lokalisierung ins Deutsche laden.
%     Wir verwenden neue deutsche Rechtschreibung mit 'ngerman'.
%
\usepackage[ngerman]{babel}
%
% 		Paket f�r Farben an verschieden Stellen. 
%     Wir definieren zus�tzliche benannte Farben an sp�terer Stelle.
%
\usepackage{color}
%\usepackage{framed}

%
% 		Das Hyperref-Package verwenden. Spezielle Optionen werden gesondert erl�utert.
%
\usepackage[
	pdftitle={F-Praktikum Josephson-Kontakt},%			 	Titel des PDF Dokuments.
	pdfauthor={PDF-Autor},%				     											Autor des PDF Dokuments.
	pdfsubject={Inhalt des PDF-Dokumentes},% 						 																										Thema des PDF Dokuments.
	pdfcreator={MiKTeX, LaTeX with hyperref and KOMA-Script},% 								Erzeuger/Kompiler des PDF Dokuments.
	pdfkeywords={Schl�sselw�rter},%																						Stichworte, die den Inhalt beschreiben.
	bookmarksopenlevel={sections},%															 								PDF-Lesezeichen bis zu Sections �ffnen.
	pdfpagemode={UseOutlines},%                                  								PDF-Lesezeichen beim �ffnen anzeigen.
	pdfdisplaydoctitle={true},%                                  								Dokumenttitel statt Dateiname in der Titelzeile anzeigen.
%	pdflanguage={german},%                                               								Sprache des Dokuments.
	pdfstartview={FitH}%																			 								Gr��e an Fensterbreite anpassen
]{hyperref}


% 
% B. EINSTELLUNGEN
% ---------------------------------------------------------------------------
%

%
%  1. Definieren von eigenen benannten Farben.
%     F�r sp�tere Verwendung in dem Dokument, definieren wir einzelne
%     benannte Farben. Sie finden bei den Links, den �berschriften und die spezielle
%			'shadecolor' bei den farbigen Boxen (mit '\begin{shaded}\end{shaded}' gesetzt) Verwendung.
%
\definecolor{LinkColor}{rgb}{0,0,0.6}
\definecolor{HeadColor}{rgb}{0,0,0.5}
\definecolor{shadecolor}{rgb}{0.85,0.95,1}

%
%  2. KOMA-Script Option, Zeilenumbruch bei Bildbeschreibungen.
%
\setcapindent{1em}

%
%  3. Stil der Kopf- und Fusszeilen.
%     Wir aktivieren mit 'headings' laufende Seiten- /Kolumnentitel.
%
\pagestyle{headings}

%
%  4. Definieren der Schriftarten/Schiftfamilien der einzelnen Bereiche.
%
\setkomafont{chapter}{\huge\scshape\color{HeadColor}}       				% Kapitel farbig, Gro� und als Kapit�lchen
\setkomafont{section}{\Large\color{HeadColor}}											% Sections farbig und gr��er als normal
\setkomafont{captionlabel}{\sffamily\bfseries\color{HeadColor}}     % Fette Beschriftungstitel, farbig und serifenlos 
\setkomafont{caption}{\sffamily}																		% serifenlose Beschriftungen (tables/figures)
\setkomafont{pagehead}{\sffamily\bfseries\color{HeadColor}}         % Kolumnentitel serifenlos, farbig und fett
\setkomafont{descriptionlabel}{\sffamily\bfseries} 									% Fette und serifenlose Beschreibungstitel
\setkomafont{pagenumber}{\sffamily\bfseries}												% Fette, serifenlose Seitenzahlen
\setkomafont{part}{\Huge\scshape\color{HeadColor}}									% 'Part'-Titel Riesig, Kapit�lchen und farbig

%
%  5. Farbeinstellungen f�r die Links im PDF Dokument.
%
\hypersetup{%
	colorlinks=true,%        Aktivieren von farbigen Links im Dokument (keine Rahmen)
	linkcolor=LinkColor,%    Farbe festlegen.
	citecolor=LinkColor,%    Farbe festlegen.
	filecolor=linkColor,%    Farbe festlegen.
	menucolor=LinkColor,%    Farbe festlegen.
	urlcolor=LinkColor,%     Farbe von URL's im Dokument.
	bookmarksnumbered=true%  �berschriftsnummerierung im PDF Inhalt anzeigen.
}

%
% D. AKTIONEN
% ---------------------------------------------------------------------------
%

%
%  1. Index erzeugen.
%
\makeindex
%
%
% E. SILBENTRENNUNG
% ---------------------------------------------------------------------------
% Nicht verwendet. Hier kann man die Silben der dem babel-package unbekannten W�rter festlegen.

\hyphenation{De-zi-mal-trenn-zeichen In-stal-la-ti-ons-as-sis-tent}


%%%%%%%%%%%%%%%%%%%%%%%%%%%%%%%%%%%%%%%%%%%%%%%%%%%%%%%%%%%%
%
%  Die Besondere Formatierung der Kapitel�berschriften (Linien dr�ber und drunter)
%
 
% 1st get a new command
\newcommand*{\ORIGchapterheadstartvskip}{}%
% 2nd save the original definition to the new command
\let\ORIGchapterheadstartvskip=\chapterheadstartvskip
% 3rd redefine the command using the saved original command
\renewcommand*{\chapterheadstartvskip}{%
  \ORIGchapterheadstartvskip
  {%
    \setlength{\parskip}{0pt}%
    \noindent\rule[.3\baselineskip]{\linewidth}{1pt}\par
  }%
}
 
% see above
\newcommand*{\ORIGchapterheadendvskip}{}%
\let\ORIGchapterheadendvskip=\chapterheadendvskip
\renewcommand*{\chapterheadendvskip}{%
  {%
    \setlength{\parskip}{0pt}%
    \noindent\rule[.3\baselineskip]{\linewidth}{1pt}\par
  }%
  \ORIGchapterheadendvskip
}
%
%  End of chapter head change
%
%%%%%%%%%%%%%%%%%%%%%%%%%%%%%%%%%%%%%%%%%%%%%%%%%%%%%%%%%%%%














%
% ===========================================================================
% EOF
%
