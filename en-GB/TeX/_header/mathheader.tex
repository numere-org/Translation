%
% mathheader
%	Packages speziell f�r den Mathesatz. 
\usepackage{array}		%
\usepackage{graphicx}
\usepackage{epsfig}
\usepackage[sumlimits,intlimits,namelimits]{amsmath}	%	AMS-MATH-Package mit der Option, die Grenzen bei Summen, Integralen und 
																											%	anderen Operatoren immer �ber und unter dem Symbol anzuzeigen.
\usepackage{amssymb,amscd}														% AMS-Packages f�r spezielle Symbole und Kommutative Diagramme
\usepackage{dsfont}																		% Symbole f�r K�rper und Mengen (C,R,N,Q,etc.)
\usepackage{mathpazo}																	% Schriftartensetup: Palatino f�r Textsatz und Pazomath f�r Mathesatz
\usepackage[scaled = 0.95]{helvet}										% Schriftartensetup: Helvetica als serifenlose Schriftart
\usepackage{courier}
\usepackage{accents}																	% Spezialpackage: Erlaubt es, sich eigene Mathe-Akzente zu definieren. Findet 
																											%	hier bei dem Befehl \vhat{} Anwendung, der ein H�tchen mit einem Pfeil 
																											%	kombiniert.
\usepackage{booktabs}																	% Erweitert die Darstellung von Tabellen
\usepackage{ziffer}																		% Kommas als Dezimaltrennzeichen

\newcommand{\bra}[1]{\left\langle #1 \right|}					% Vordefinerter, skalierender Bra-Vektor
\newcommand{\tbra}[1]{\langle #1|}										% Nicht-Skalierender Bra-Vektor. Speziell f�r den Textsatz, damit die Zeilen 
																											%	nicht so �berstreckt werden.
\newcommand{\ket}[1]{\left| #1\right\rangle}					% Selbiges f�r Ket
\newcommand{\tket}[1]{| #1 \rangle}										% Selbiges f�r Ket
\newcommand{\skal}[2]{\left\langle #1 | #2 \right\rangle}	%Physikalische Bra-Ket-Skalaproduktschreibweise
\newcommand{\mskal}[2]{\left\langle #1 , #2 \right\rangle} %Mathematisches Skalarprodukt	
\newcommand{\tskal}[2]{\langle #1|#2\rangle}					% Nicht-Skalierendes Skalarprodukt. F�r den Textsatz.
\newcommand{\EW}[1]{\left\langle #1 \right\rangle}		% Skalierende Eigenwertklammern
\newcommand{\ew}[1]{\langle #1\rangle}								% Nicht-Skalierende Eigenwertklammern
\newcommand{\dx}{\:\text{\normalfont{d}}}							% Befehl f�r das 'd' beim Integral. Befehl beinh�lt Abstand vor dem und ein 
																											% Nicht-Kursives d.
\newcommand{\e}[1]{\:\text{\normalfont{e}}^{#1}}			% Befehl f�r die exp-Funktion. Ebenfalls bereits aufrecht, und mit 
																											% vorangehendem Abstand.
\newcommand{\eps}{\varepsilon}												% Abk�rzung f�r das \varepsilon
%  Abk�rzunge f�r Mathematische K�rper und Mengen
\newcommand{\N}{\mathds N}
\newcommand{\Z}{\mathds Z}
\newcommand{\R}{\mathds R}
\newcommand{\Q}{\mathds Q}
\newcommand{\C}{\mathds C}
\newcommand{\K}{\mathds K}
\newcommand{\V}{\mathds V}
\newcommand{\I}{\Im}

\newcommand{\prt}[2]{\frac{\partial #1}{\partial #2}}	% Differenzenquotient, Partielle Ableitung
\newcommand{\diff}[2]{\frac{\text{d} #1}{\text{d} #2}} % Differenzenquotient, totale Ableitung: df/dx
\newcommand{\Diff}[1]{\frac{\text{d}}{\text{d} #1}}		% Differenzenquotient, allerdings ohne Einbindung der abgel. Funktion: d/dx
\newcommand{\svec}[1]{\begin{pmatrix} #1 \end{pmatrix}} % Abk�rzung f�r Spaltenvektoren
\newcommand{\detm}[1]{\begin{vmatrix} #1 \end{vmatrix}}
\newcommand{\abs}[1]{\left|#1\right|}									%	Skalierende Betragsstriche
\newcommand{\norm}[1]{\left\|#1\right\|}							%	Skalierende Norm-Striche: ||Vektor||
\newcommand{\trot}{\text{rot}\:}											% Rotations-Operator: \trot\vec E = 0
\newcommand{\chem}[2]{\shortstack[r]{\tiny{${#1}$}\\\scriptsize{$_{#2}$}}} %F�r Isotopenschreibweise von Elementen
\newcommand{\Angstrom}{\text{A}\hspace{-6.55pt}^\text{�}}	% Eigener Angstrom-Befehl
\newcommand{\wh}[1]{\hat{#1}}													% Abk�rzung f�r den \hat-Befehl
\newcommand{\vhat}[1]{\accentset{\hspace{1.6pt}\text{\fontsize{4.5pt}{4pt}{$\boldsymbol\wedge$}}\hspace{-3.9pt}\text{\fontsize{5pt}{4pt}{$\boldsymbol{\rightarrow}$}}}{#1}\hspace{0.4pt}}	% Spezieller Accent, der H�tchen mit Vektorpfeil kombiniert
\newcommand{\tx}[1]{_{\text{#1}}}							% Befehl f�r aufrecht gesetzten Index

%  Der \vldt{} Befehl: Zeichet eine Linie, die das Argument als eine Marginalie auf dem Rand und �ber dem Strich darstellt.
\newcommand{\vldt}[1]{\rule{0mm}{8mm}\textsf{\small{\textcolor{HeadColor}{$\blacktriangledown$ \textbf{Vorlesung vom #1}}}}
\marginline{\fbox{\textsf{\textbf{\small{\textcolor{HeadColor}{#1}}}}}}\newline\rule[.75\baselineskip]{\linewidth}{0.5pt}}
%  Abk�rzung f�r den \displaybreak[1]. Mit diesem Befehl kann man 'align'-Umgebungen beim Seitenende gezielt umbrechen.
\newcommand{\br}{\displaybreak[1]}
%----------Textkommandos: Abk�rzungen und Formatierungen f�r Begriffe und Formulierungen-----------------
\newcommand{\DEF}[1]{\subsection[Definition: #1]{Definition (#1):}}
\newcommand{\KOR}[1]{\subsection[Korollar: #1]{Korollar (#1):}}
\newcommand{\LEMMA}[1]{\subsection[Lemma: #1]{Lemma (#1):}}
\newcommand{\BEM}{\paragraph{Bemerkung:}}
\newcommand{\BSP}{\paragraph{\textcolor{HeadColor}{Beispiel.}}}
\newcommand{\SATZ}[1]{\subsection[Satz: #1]{\textsc{Satz (#1):}}}

%----------Umgebung f�r eine alphabetisch nummerierte Liste------
\newcounter{ale}
\newcommand{\abc}{\item[\alph{ale})]\stepcounter{ale}}
\newenvironment{liste}{\begin{itemize}}{\end{itemize}}
\newcommand{\aliste}{\begin{liste} \setcounter{ale}{1}}
\newcommand{\zliste}{\end{liste}}
\newenvironment{abcliste}{\aliste}{\zliste}
%	Setzen mit '\aliste\abc Text \zliste'
%----------------------------------------
