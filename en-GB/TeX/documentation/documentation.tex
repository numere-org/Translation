\documentclass[DIV=17, parskip=half]{scrreprt}
% Main file for the documentation file D:/Software/NumeRe/scripts/cheat_sheet.nscr
\usepackage[headsepline]{scrlayer-scrpage}% activates pagestyle scrheadings automatically
%\usepackage[ngerman]{babel}
\input{D:/Software/NumeRe/save/docs/numereheader}
\definecolor{cLink}{RGB}{0,0,128}
\usepackage[linkcolor=cLink,colorlinks=true]{hyperref}
\automark{chapter}
\newcommand{\nrdesc}{Description}
\newcommand{\nrexample}{Example}

\ohead{\headmark}
\chead{}
\ihead{NumeRe: Documentation}
\ifoot{Free numerical software}
\ofoot{Get it at: www.numere.org}
\pagestyle{scrheadings}
\subject{NumeRe: Framework for Numerical Computations}
\title{Documentation}
\subtitle{NumeRe v1.1.7 >>Release Candidate<<}
\date{}
\author{\small Provided to you by NumeRe.org}
\begin{document}
    \maketitle
% 	\begin{abstract}
% 		\textbf{Even the longest way begins with the first step} once said Lao-Tse, Chinese philosopher in the 6th century before Christ. And of course this statement is necessarily true. But what does it say?
% 
% 		In essence, it is about starting in the first place, even with difficult topics. And because we know that it can be really difficult to find the direction for the first step, we help you here with the most important basics in NumeRe.
% 
% 		For all topics it can be helpful to have a look at the NumeRe documentation, which you can display with [F1], the context menu or with the command help. We would also like to recommend the search function in the toolbar if you get stuck.
% 	\end{abstract}
	\tableofcontents
	\addsec{Copyrights}
		\paragraph{Copyright \copyright\ 2013-2024, Erik H\"anel \emph{et al.}} 
		NumeRe has been licensed under the terms of the GNU General Public License v3, available at \href{http://www.gnu.org/licenses/gpl.html}{$\Rightarrow$ GPL}
		\paragraph{External Dependencies and Licenses} A list of all external dependencies and their associate licenses may be found in the file \verb!THIRD_PARTY.licences! in the installation root directory.
	\chapter{Basics}
		Get to know the basics of NumeRe here. If you are completely new here, you will also find a section on the first steps in this chapter. In it, we show you how to use NumeRe using simple lines of code.
		\input{articles/main.tex}
		\input{articles/firststeps.tex}
		\input{articles/syntax.tex}
		\input{articles/expression.tex}
		\input{articles/functions.tex}
		\input{articles/find.tex}
		\input{articles/list.tex}
		\input{articles/get.tex}
		\input{articles/set.tex}
		\input{articles/quit.tex}
	
	\chapter{Interface \& Editor}
		This chapter gives you an overview of the graphical user interface, especially the editor. Learn how to write and optimize NumeRe code as efficiently as possible. You will also find a description of the debugger, which will help you with problems in the code.
		\input{articles/editor.tex}
		\input{articles/codeanalyzer.tex}
		\input{articles/refactoring.tex}
		\input{articles/versioncontrol.tex}
		\input{articles/console.tex}
		\input{articles/debugger.tex}
		\input{articles/filetree.tex}
		\input{articles/symboltree.tex}
		\input{articles/graphviewer.tex}
		\input{articles/history.tex}
		
	\chapter{Variables \& Data Structures}
		Get to know the central data structures of NumeRe and their interactions. Central to this are the tables with their specialized methods, which simplify data management and processing many times over.
		\input{articles/variables.tex}
		\input{articles/string.tex}
		\input{articles/rekurs_oprt.tex}
		\input{articles/multiresult.tex}
		\input{articles/cache.tex}
		\input{articles/cluster.tex}
		\input{articles/objects.tex}
		\input{articles/new.tex}
		\input{articles/show.tex}
	
	\chapter{File System}
		This chapter gives you an insight into the file system, how data can be loaded and written, how you can read in any files, how files can be copied and moved automatically and how you can open archive files.
		\input{articles/data.tex}
		\input{articles/edit.tex}
		\input{articles/load_save.tex}
		\input{articles/read_write.tex}
		\input{articles/data_and_fileops.tex}
		\input{articles/explorer.tex}
		\input{articles/pack.tex}
	
	\chapter{Plotting}
		From simple line and dot plots to three-dimensional vector plots, NumeRe has a considerable number of different styles and variants from which you can choose for your data. Learn here what these are and how you can access them.
		\input{articles/plot.tex}
		\input{articles/plot3d.tex}
		\input{articles/2dplots.tex}
		\input{articles/3dplots.tex}
		\input{articles/drawing.tex}
		\input{articles/3ddrawing.tex}
		\input{articles/gradient.tex}
		\input{articles/gradient3d.tex}
		\input{articles/vect.tex}
		\input{articles/vect3d.tex}
		\input{articles/compose.tex}
		\input{articles/subplot.tex}
		\input{articles/plotoptions.tex}
		\input{articles/tickformatting.tex}
		\input{articles/color.tex}
		\input{articles/coords.tex}
		\input{articles/linestyles.tex}
	
	\chapter{Data Processing}
		A significant proportion of the functionalities are those that focus on data processing. This includes sorting, creating histograms, audio data processing, Fourier transformations, fitting and much more.
		\input{articles/sort.tex}
		\input{articles/stats.tex}
		\input{articles/histogramm.tex}
		\input{articles/units.tex}
		\input{articles/audio.tex}
		\input{articles/imread.tex}
		\input{articles/eval.tex}
		\input{articles/datagrid.tex}
		\input{articles/datamodification.tex}
		\input{articles/regularize.tex}
		\input{articles/rotate.tex}
		\input{articles/detect.tex}
		\input{articles/extrema.tex}
		\input{articles/zeroes.tex}
		\input{articles/fft.tex}
		\input{articles/stfa.tex}
		\input{articles/fwt.tex}
		\input{articles/fitting.tex}
		\input{articles/fitting_restrictions.tex}
		\input{articles/fitting_chimap.tex}
		\input{articles/pulse.tex}
		\input{articles/random.tex}
	
	\chapter{Mathematical Operations}
		This chapter deals with more extensive mathematical operations such as integrating, differentiating, matrix operations and similar.
		\input{articles/integrate.tex}
		\input{articles/diff.tex}
		\input{articles/taylor.tex}
		\input{articles/spline.tex}
		\input{articles/matop.tex}
		\input{articles/matop_functions.tex}
		\input{articles/pso.tex}
		\input{articles/odesolve.tex}
		
	\chapter{Control Flow}
		Understanding the control flow blocks is essential on the path to advanced automation. Get to know the six central control flow blocks here.
		\input{articles/flow_ctrl.tex}
		\input{articles/if_cond.tex}
		\input{articles/for_loop.tex}
		\input{articles/while_loop.tex}
		\input{articles/switch.tex}
		\input{articles/try_catch.tex}
		\input{articles/abort.tex}
	
	\chapter{Automation}
		From simple scripts to complex procedures, you will find everything your heart desires in terms of automation. You will also learn how to define file-specific constants and how to bundle your solutions into packages and plugins so that others can use them.
		\input{articles/script.tex}
		\input{articles/prompt.tex}
		\input{articles/define.tex}
		\input{articles/declare.tex}
		\input{articles/include.tex}
		\input{articles/procedure.tex}
		\input{articles/procedure_commands.tex}
		\input{articles/install.tex}
		\input{articles/plugins.tex}
		\input{articles/progress.tex}
	
	\chapter{Graphical User Interfaces}
		If you want to put the finishing touches to your automations, the graphical user interfaces are just the thing for you. Use simple dialogs for quick user interactions and complex window layouts for the complete look 'n feel of entire applications.
		\input{articles/dialog.tex}
		\input{articles/window.tex}
	
	\chapter{Networking and Databases}
		Working with remote data sources is central in modern data analytics without any alternative and will be described in this chapter.
		\input{articles/url.tex}
		\input{articles/mail.tex}
		\input{articles/database.tex}
		
	\chapter{Miscellaneous}
		In this chapter you will find additional commands and functionalities that have not yet been covered in the other chapters.
		\input{articles/execute.tex}
		\input{articles/additionalcommands.tex}
		\input{articles/language.tex}
		\input{articles/latex.tex}
		\documentclass[DIV=17, parskip=half]{scrreprt}
% Main file for the documentation file D:/Software/NumeRe/scripts/cheat_sheet.nscr
\usepackage[headsepline]{scrlayer-scrpage}% activates pagestyle scrheadings automatically
\usepackage[ngerman]{babel}
\input{D:/Software/NumeRe/save/docs/numereheader}
\definecolor{cLink}{RGB}{0,0,128}
\usepackage[linkcolor=cLink,colorlinks=true]{hyperref}
\automark{chapter}
\newcommand{\nrdesc}{Beschreibung}
\newcommand{\nrexample}{Beispiel}

\ohead{\headmark}
\chead{}
\ihead{NumeRe: Dokumentation}
\ifoot{Free numerical software}
\ofoot{Get it at: www.numere.org}
\pagestyle{scrheadings}
\subject{NumeRe: Framework f\"ur Numerische Rechnungen}
\title{Dokumentation}
\subtitle{NumeRe v1.1.7 >>Release Candidate<<}
\date{}
\author{\small Provided to you by NumeRe.org}
\begin{document}
    \maketitle
% 	\begin{abstract}
% 		\textbf{Even the longest way begins with the first step} once said Lao-Tse, Chinese philosopher in the 6th century before Christ. And of course this statement is necessarily true. But what does it say?
% 
% 		In essence, it is about starting in the first place, even with difficult topics. And because we know that it can be really difficult to find the direction for the first step, we help you here with the most important basics in NumeRe.
% 
% 		For all topics it can be helpful to have a look at the NumeRe documentation, which you can display with [F1], the context menu or with the command help. We would also like to recommend the search function in the toolbar if you get stuck.
% 	\end{abstract}
	\tableofcontents
	\addsec{Copyrights}
		\paragraph{Copyright \copyright\ 2013-2024 Erik H\"anel \emph{et al.}} 
		NumeRe ist lizensiert unter der GNU General Public License v3, verf\"ugbar unter \href{http://www.gnu.org/licenses/gpl.html}{$\Rightarrow$ GPL}
		\paragraph{Externe Abh\"angigkeiten und Lizenzen} Eine Auflistung aller externen Abh\"angigkeiten und der zugeh\"origen Lizenzen kann in der Datei \verb!THIRD_PARTY.licenses! im Installationsverzeichnis gefunden werden.
	\chapter{Grundlagen}
		Lerne hier die Grundlagen von NumeRe kennen. Wenn du komplett neu hier bist, dann findest du hier in diesem Kapitel auch einen Abschnitt zu den ersten Schritten. Darin zeigen wir dir anhand einfacher Code-Zeilen, wie du NumeRe verwenden kannst.
		\input{articles/main.tex}
		\input{articles/firststeps.tex}
		\input{articles/syntax.tex}
		\input{articles/expression.tex}
		\input{articles/functions.tex}
		\input{articles/find.tex}
		\input{articles/list.tex}
		\input{articles/get.tex}
		\input{articles/set.tex}
		\input{articles/quit.tex}
	
	\chapter{Interface \& Editor}
		Dieses Kapitel gibt dir einen \"Uberblick \"uber die graphische Benutzerberfl\"ache, insbesondere den Editor. Lerne hier, wie du NumeRe-Code so effizient wie m\"oglich schreiben und optimieren kannst. Au\ss erdem findest du hier einen Beschreibung des Debuggers, der dich bei Problemen im Code unterst\"utzt.
		\input{articles/editor.tex}
		\input{articles/codeanalyzer.tex}
		\input{articles/refactoring.tex}
		\input{articles/versioncontrol.tex}
		\input{articles/console.tex}
		\input{articles/debugger.tex}
		\input{articles/filetree.tex}
		\input{articles/symboltree.tex}
		\input{articles/graphviewer.tex}
		\input{articles/history.tex}
		
	\chapter{Variablen \& Datenstrukturen}
		Lerne hier die zentralen Datenstrukturen von NumeRe und deren Interaktionen kennen. Zentral sind hier die Tabellen mit ihren spezialisierten Methoden, welche die Datenverwaltung und -verarbeitung um ein Vielfaches vereinfachen.
		\input{articles/variables.tex}
		\input{articles/string.tex}
		\input{articles/rekurs_oprt.tex}
		\input{articles/multiresult.tex}
		\input{articles/cache.tex}
		\input{articles/cluster.tex}
		\input{articles/new.tex}
		\input{articles/show.tex}
	
	\chapter{Dateisystem}
		Dieses Kapitel gibt dir einen Einblick in das Dateisystem, wie Daten geladen und geschrieben werden k\"onnen, wie man beliebige Dateien einlesen kann, wie Dateien automatisiert kopiert und verschoben werden k\"onnen und wie du Archivdateien \"offnen kannst.
		\input{articles/data.tex}
		\input{articles/edit.tex}
		\input{articles/load_save.tex}
		\input{articles/read_write.tex}
		\input{articles/data_and_fileops.tex}
		\input{articles/explorer.tex}
		\input{articles/pack.tex}
	
	\chapter{Plotting}
		NumeRe beherrscht von einfachen Linien- und Punktplots bis hin zu dreidimensionalen Vektorplots eine beachtliche Anzahl an unterschiedliche Stilen und Varianten, aus denen du f\"ur deine Daten w\"ahlen kannst. Lerne hier, welche das sind und wie du darauf zugreifen kannst.
		\input{articles/plot.tex}
		\input{articles/plot3d.tex}
		\input{articles/2dplots.tex}
		\input{articles/3dplots.tex}
		\input{articles/drawing.tex}
		\input{articles/3ddrawing.tex}
		\input{articles/gradient.tex}
		\input{articles/gradient3d.tex}
		\input{articles/vect.tex}
		\input{articles/vect3d.tex}
		\input{articles/compose.tex}
		\input{articles/subplot.tex}
		\input{articles/plotoptions.tex}
		\input{articles/tickformatting.tex}
		\input{articles/color.tex}
		\input{articles/coords.tex}
		\input{articles/linestyles.tex}
	
	\chapter{Datenverarbeitung}
		Ein bedeutender Anteil der Funktionalit\"aten sind solche, die sich auf Datenverarbeitung fokussieren. Dazu geh\"ort Sortieren, Erstellen von Histogrammen, Audiodatenverarbeitung, Fouriertransformationen, Fitten und vieles weiteres.
		\input{articles/sort.tex}
		\input{articles/stats.tex}
		\input{articles/histogramm.tex}
		\input{articles/units.tex}
		\input{articles/audio.tex}
		\input{articles/imread.tex}
		\input{articles/eval.tex}
		\input{articles/datagrid.tex}
		\input{articles/datamodification.tex}
		\input{articles/regularize.tex}
		\input{articles/rotate.tex}
		\input{articles/detect.tex}
		\input{articles/extrema.tex}
		\input{articles/zeroes.tex}
		\input{articles/fft.tex}
		\input{articles/stfa.tex}
		\input{articles/fwt.tex}
		\input{articles/fitting.tex}
		\input{articles/fitting_restrictions.tex}
		\input{articles/fitting_chimap.tex}
		\input{articles/pulse.tex}
		\input{articles/random.tex}
	
	\chapter{Mathematische Operationen}
		Dieses Kapitel dreht sich um umfangreichere mathematische Operationen wie Integrieren, Differenzieren, Matrix-Operationen und Vergleichbares.
		\input{articles/integrate.tex}
		\input{articles/diff.tex}
		\input{articles/taylor.tex}
		\input{articles/spline.tex}
		\input{articles/matop.tex}
		\input{articles/matop_functions.tex}
		\input{articles/pso.tex}
		\input{articles/odesolve.tex}
		
	\chapter{Kontrollfluss}
		Ein Verst\"andnis der Kontrollflussbl\"ocke ist auf dem Weg zur fortgeschrittenen Automatisierung unabdingbar. Lerne hier die sechs zentralen Kontrollflussbl\"ocke kennen.
		\input{articles/flow_ctrl.tex}
		\input{articles/if_cond.tex}
		\input{articles/for_loop.tex}
		\input{articles/while_loop.tex}
		\input{articles/switch.tex}
		\input{articles/try_catch.tex}
		\input{articles/abort.tex}
	
	\chapter{Automatisierung}
		Von einfachen Scripten bis hin zu komplexen Prozeduren findest du hier alles, was dein Herz bez\"uglich Automatisierung begehrt. Lerne hier auch, wie du dateispezifische Konstanten definieren kannst und wie du deine L\"osungen in Packages und Plugins b\"undeln kannst, so dass andere sie verwenden k\"onnen.
		\input{articles/script.tex}
		\input{articles/prompt.tex}
		\input{articles/define.tex}
		\input{articles/declare.tex}
		\input{articles/include.tex}
		\input{articles/procedure.tex}
		\input{articles/procedure_commands.tex}
		\input{articles/install.tex}
		\input{articles/plugins.tex}
		\input{articles/progress.tex}
	
	\chapter{Graphische User Interfaces}
		Wenn du deinen Automatisierungen noch den finalen Feinschliff verpassen willst, bist du bei den graphischen User Interfaces genau richtig. Verwende einfache Dialoge f\"ur schnelle Benutzerinteraktionen und komplexe Window-Layouts f\"ur das komplette Look 'n Feel ausgereifter Applikationen.
		\input{articles/dialog.tex}
		\input{articles/window.tex}
	
	\chapter{Netzwerk und Datenbanken}
		Das Arbeiten mit Remote-Datenquellen ist in der modernen Datenanalyse unabdingbar und wird in diesem Kapitel erl\"autert.
		\input{articles/url.tex}
		\input{articles/mail.tex}
		\input{articles/database.tex}
		
	\chapter{Verschiedenes}
		In diesem Kapitel findest du weitere Kommandos und Funktionalit\"aten, die noch nicht in den anderen Kapiteln behandelt wurden.
		\input{articles/execute.tex}
		\input{articles/additionalcommands.tex}
		\input{articles/language.tex}
		\input{articles/latex.tex}
		\documentclass[DIV=17, parskip=half]{scrreprt}
% Main file for the documentation file D:/Software/NumeRe/scripts/cheat_sheet.nscr
\usepackage[headsepline]{scrlayer-scrpage}% activates pagestyle scrheadings automatically
\usepackage[ngerman]{babel}
\input{D:/Software/NumeRe/save/docs/numereheader}
\definecolor{cLink}{RGB}{0,0,128}
\usepackage[linkcolor=cLink,colorlinks=true]{hyperref}
\automark{chapter}
\newcommand{\nrdesc}{Beschreibung}
\newcommand{\nrexample}{Beispiel}

\ohead{\headmark}
\chead{}
\ihead{NumeRe: Dokumentation}
\ifoot{Free numerical software}
\ofoot{Get it at: www.numere.org}
\pagestyle{scrheadings}
\subject{NumeRe: Framework f\"ur Numerische Rechnungen}
\title{Dokumentation}
\subtitle{NumeRe v1.1.7 >>Release Candidate<<}
\date{}
\author{\small Provided to you by NumeRe.org}
\begin{document}
    \maketitle
% 	\begin{abstract}
% 		\textbf{Even the longest way begins with the first step} once said Lao-Tse, Chinese philosopher in the 6th century before Christ. And of course this statement is necessarily true. But what does it say?
% 
% 		In essence, it is about starting in the first place, even with difficult topics. And because we know that it can be really difficult to find the direction for the first step, we help you here with the most important basics in NumeRe.
% 
% 		For all topics it can be helpful to have a look at the NumeRe documentation, which you can display with [F1], the context menu or with the command help. We would also like to recommend the search function in the toolbar if you get stuck.
% 	\end{abstract}
	\tableofcontents
	\addsec{Copyrights}
		\paragraph{Copyright \copyright\ 2013-2024 Erik H\"anel \emph{et al.}} 
		NumeRe ist lizensiert unter der GNU General Public License v3, verf\"ugbar unter \href{http://www.gnu.org/licenses/gpl.html}{$\Rightarrow$ GPL}
		\paragraph{Externe Abh\"angigkeiten und Lizenzen} Eine Auflistung aller externen Abh\"angigkeiten und der zugeh\"origen Lizenzen kann in der Datei \verb!THIRD_PARTY.licenses! im Installationsverzeichnis gefunden werden.
	\chapter{Grundlagen}
		Lerne hier die Grundlagen von NumeRe kennen. Wenn du komplett neu hier bist, dann findest du hier in diesem Kapitel auch einen Abschnitt zu den ersten Schritten. Darin zeigen wir dir anhand einfacher Code-Zeilen, wie du NumeRe verwenden kannst.
		\input{articles/main.tex}
		\input{articles/firststeps.tex}
		\input{articles/syntax.tex}
		\input{articles/expression.tex}
		\input{articles/functions.tex}
		\input{articles/find.tex}
		\input{articles/list.tex}
		\input{articles/get.tex}
		\input{articles/set.tex}
		\input{articles/quit.tex}
	
	\chapter{Interface \& Editor}
		Dieses Kapitel gibt dir einen \"Uberblick \"uber die graphische Benutzerberfl\"ache, insbesondere den Editor. Lerne hier, wie du NumeRe-Code so effizient wie m\"oglich schreiben und optimieren kannst. Au\ss erdem findest du hier einen Beschreibung des Debuggers, der dich bei Problemen im Code unterst\"utzt.
		\input{articles/editor.tex}
		\input{articles/codeanalyzer.tex}
		\input{articles/refactoring.tex}
		\input{articles/versioncontrol.tex}
		\input{articles/console.tex}
		\input{articles/debugger.tex}
		\input{articles/filetree.tex}
		\input{articles/symboltree.tex}
		\input{articles/graphviewer.tex}
		\input{articles/history.tex}
		
	\chapter{Variablen \& Datenstrukturen}
		Lerne hier die zentralen Datenstrukturen von NumeRe und deren Interaktionen kennen. Zentral sind hier die Tabellen mit ihren spezialisierten Methoden, welche die Datenverwaltung und -verarbeitung um ein Vielfaches vereinfachen.
		\input{articles/variables.tex}
		\input{articles/string.tex}
		\input{articles/rekurs_oprt.tex}
		\input{articles/multiresult.tex}
		\input{articles/cache.tex}
		\input{articles/cluster.tex}
		\input{articles/new.tex}
		\input{articles/show.tex}
	
	\chapter{Dateisystem}
		Dieses Kapitel gibt dir einen Einblick in das Dateisystem, wie Daten geladen und geschrieben werden k\"onnen, wie man beliebige Dateien einlesen kann, wie Dateien automatisiert kopiert und verschoben werden k\"onnen und wie du Archivdateien \"offnen kannst.
		\input{articles/data.tex}
		\input{articles/edit.tex}
		\input{articles/load_save.tex}
		\input{articles/read_write.tex}
		\input{articles/data_and_fileops.tex}
		\input{articles/explorer.tex}
		\input{articles/pack.tex}
	
	\chapter{Plotting}
		NumeRe beherrscht von einfachen Linien- und Punktplots bis hin zu dreidimensionalen Vektorplots eine beachtliche Anzahl an unterschiedliche Stilen und Varianten, aus denen du f\"ur deine Daten w\"ahlen kannst. Lerne hier, welche das sind und wie du darauf zugreifen kannst.
		\input{articles/plot.tex}
		\input{articles/plot3d.tex}
		\input{articles/2dplots.tex}
		\input{articles/3dplots.tex}
		\input{articles/drawing.tex}
		\input{articles/3ddrawing.tex}
		\input{articles/gradient.tex}
		\input{articles/gradient3d.tex}
		\input{articles/vect.tex}
		\input{articles/vect3d.tex}
		\input{articles/compose.tex}
		\input{articles/subplot.tex}
		\input{articles/plotoptions.tex}
		\input{articles/tickformatting.tex}
		\input{articles/color.tex}
		\input{articles/coords.tex}
		\input{articles/linestyles.tex}
	
	\chapter{Datenverarbeitung}
		Ein bedeutender Anteil der Funktionalit\"aten sind solche, die sich auf Datenverarbeitung fokussieren. Dazu geh\"ort Sortieren, Erstellen von Histogrammen, Audiodatenverarbeitung, Fouriertransformationen, Fitten und vieles weiteres.
		\input{articles/sort.tex}
		\input{articles/stats.tex}
		\input{articles/histogramm.tex}
		\input{articles/units.tex}
		\input{articles/audio.tex}
		\input{articles/imread.tex}
		\input{articles/eval.tex}
		\input{articles/datagrid.tex}
		\input{articles/datamodification.tex}
		\input{articles/regularize.tex}
		\input{articles/rotate.tex}
		\input{articles/detect.tex}
		\input{articles/extrema.tex}
		\input{articles/zeroes.tex}
		\input{articles/fft.tex}
		\input{articles/stfa.tex}
		\input{articles/fwt.tex}
		\input{articles/fitting.tex}
		\input{articles/fitting_restrictions.tex}
		\input{articles/fitting_chimap.tex}
		\input{articles/pulse.tex}
		\input{articles/random.tex}
	
	\chapter{Mathematische Operationen}
		Dieses Kapitel dreht sich um umfangreichere mathematische Operationen wie Integrieren, Differenzieren, Matrix-Operationen und Vergleichbares.
		\input{articles/integrate.tex}
		\input{articles/diff.tex}
		\input{articles/taylor.tex}
		\input{articles/spline.tex}
		\input{articles/matop.tex}
		\input{articles/matop_functions.tex}
		\input{articles/pso.tex}
		\input{articles/odesolve.tex}
		
	\chapter{Kontrollfluss}
		Ein Verst\"andnis der Kontrollflussbl\"ocke ist auf dem Weg zur fortgeschrittenen Automatisierung unabdingbar. Lerne hier die sechs zentralen Kontrollflussbl\"ocke kennen.
		\input{articles/flow_ctrl.tex}
		\input{articles/if_cond.tex}
		\input{articles/for_loop.tex}
		\input{articles/while_loop.tex}
		\input{articles/switch.tex}
		\input{articles/try_catch.tex}
		\input{articles/abort.tex}
	
	\chapter{Automatisierung}
		Von einfachen Scripten bis hin zu komplexen Prozeduren findest du hier alles, was dein Herz bez\"uglich Automatisierung begehrt. Lerne hier auch, wie du dateispezifische Konstanten definieren kannst und wie du deine L\"osungen in Packages und Plugins b\"undeln kannst, so dass andere sie verwenden k\"onnen.
		\input{articles/script.tex}
		\input{articles/prompt.tex}
		\input{articles/define.tex}
		\input{articles/declare.tex}
		\input{articles/include.tex}
		\input{articles/procedure.tex}
		\input{articles/procedure_commands.tex}
		\input{articles/install.tex}
		\input{articles/plugins.tex}
		\input{articles/progress.tex}
	
	\chapter{Graphische User Interfaces}
		Wenn du deinen Automatisierungen noch den finalen Feinschliff verpassen willst, bist du bei den graphischen User Interfaces genau richtig. Verwende einfache Dialoge f\"ur schnelle Benutzerinteraktionen und komplexe Window-Layouts f\"ur das komplette Look 'n Feel ausgereifter Applikationen.
		\input{articles/dialog.tex}
		\input{articles/window.tex}
	
	\chapter{Netzwerk und Datenbanken}
		Das Arbeiten mit Remote-Datenquellen ist in der modernen Datenanalyse unabdingbar und wird in diesem Kapitel erl\"autert.
		\input{articles/url.tex}
		\input{articles/mail.tex}
		\input{articles/database.tex}
		
	\chapter{Verschiedenes}
		In diesem Kapitel findest du weitere Kommandos und Funktionalit\"aten, die noch nicht in den anderen Kapiteln behandelt wurden.
		\input{articles/execute.tex}
		\input{articles/additionalcommands.tex}
		\input{articles/language.tex}
		\input{articles/latex.tex}
		\documentclass[DIV=17, parskip=half]{scrreprt}
% Main file for the documentation file D:/Software/NumeRe/scripts/cheat_sheet.nscr
\usepackage[headsepline]{scrlayer-scrpage}% activates pagestyle scrheadings automatically
\usepackage[ngerman]{babel}
\input{D:/Software/NumeRe/save/docs/numereheader}
\definecolor{cLink}{RGB}{0,0,128}
\usepackage[linkcolor=cLink,colorlinks=true]{hyperref}
\automark{chapter}
\newcommand{\nrdesc}{Beschreibung}
\newcommand{\nrexample}{Beispiel}

\ohead{\headmark}
\chead{}
\ihead{NumeRe: Dokumentation}
\ifoot{Free numerical software}
\ofoot{Get it at: www.numere.org}
\pagestyle{scrheadings}
\subject{NumeRe: Framework f\"ur Numerische Rechnungen}
\title{Dokumentation}
\subtitle{NumeRe v1.1.7 >>Release Candidate<<}
\date{}
\author{\small Provided to you by NumeRe.org}
\begin{document}
    \maketitle
% 	\begin{abstract}
% 		\textbf{Even the longest way begins with the first step} once said Lao-Tse, Chinese philosopher in the 6th century before Christ. And of course this statement is necessarily true. But what does it say?
% 
% 		In essence, it is about starting in the first place, even with difficult topics. And because we know that it can be really difficult to find the direction for the first step, we help you here with the most important basics in NumeRe.
% 
% 		For all topics it can be helpful to have a look at the NumeRe documentation, which you can display with [F1], the context menu or with the command help. We would also like to recommend the search function in the toolbar if you get stuck.
% 	\end{abstract}
	\tableofcontents
	\addsec{Copyrights}
		\paragraph{Copyright \copyright\ 2013-2024 Erik H\"anel \emph{et al.}} 
		NumeRe ist lizensiert unter der GNU General Public License v3, verf\"ugbar unter \href{http://www.gnu.org/licenses/gpl.html}{$\Rightarrow$ GPL}
		\paragraph{Externe Abh\"angigkeiten und Lizenzen} Eine Auflistung aller externen Abh\"angigkeiten und der zugeh\"origen Lizenzen kann in der Datei \verb!THIRD_PARTY.licenses! im Installationsverzeichnis gefunden werden.
	\chapter{Grundlagen}
		Lerne hier die Grundlagen von NumeRe kennen. Wenn du komplett neu hier bist, dann findest du hier in diesem Kapitel auch einen Abschnitt zu den ersten Schritten. Darin zeigen wir dir anhand einfacher Code-Zeilen, wie du NumeRe verwenden kannst.
		\input{articles/main.tex}
		\input{articles/firststeps.tex}
		\input{articles/syntax.tex}
		\input{articles/expression.tex}
		\input{articles/functions.tex}
		\input{articles/find.tex}
		\input{articles/list.tex}
		\input{articles/get.tex}
		\input{articles/set.tex}
		\input{articles/quit.tex}
	
	\chapter{Interface \& Editor}
		Dieses Kapitel gibt dir einen \"Uberblick \"uber die graphische Benutzerberfl\"ache, insbesondere den Editor. Lerne hier, wie du NumeRe-Code so effizient wie m\"oglich schreiben und optimieren kannst. Au\ss erdem findest du hier einen Beschreibung des Debuggers, der dich bei Problemen im Code unterst\"utzt.
		\input{articles/editor.tex}
		\input{articles/codeanalyzer.tex}
		\input{articles/refactoring.tex}
		\input{articles/versioncontrol.tex}
		\input{articles/console.tex}
		\input{articles/debugger.tex}
		\input{articles/filetree.tex}
		\input{articles/symboltree.tex}
		\input{articles/graphviewer.tex}
		\input{articles/history.tex}
		
	\chapter{Variablen \& Datenstrukturen}
		Lerne hier die zentralen Datenstrukturen von NumeRe und deren Interaktionen kennen. Zentral sind hier die Tabellen mit ihren spezialisierten Methoden, welche die Datenverwaltung und -verarbeitung um ein Vielfaches vereinfachen.
		\input{articles/variables.tex}
		\input{articles/string.tex}
		\input{articles/rekurs_oprt.tex}
		\input{articles/multiresult.tex}
		\input{articles/cache.tex}
		\input{articles/cluster.tex}
		\input{articles/new.tex}
		\input{articles/show.tex}
	
	\chapter{Dateisystem}
		Dieses Kapitel gibt dir einen Einblick in das Dateisystem, wie Daten geladen und geschrieben werden k\"onnen, wie man beliebige Dateien einlesen kann, wie Dateien automatisiert kopiert und verschoben werden k\"onnen und wie du Archivdateien \"offnen kannst.
		\input{articles/data.tex}
		\input{articles/edit.tex}
		\input{articles/load_save.tex}
		\input{articles/read_write.tex}
		\input{articles/data_and_fileops.tex}
		\input{articles/explorer.tex}
		\input{articles/pack.tex}
	
	\chapter{Plotting}
		NumeRe beherrscht von einfachen Linien- und Punktplots bis hin zu dreidimensionalen Vektorplots eine beachtliche Anzahl an unterschiedliche Stilen und Varianten, aus denen du f\"ur deine Daten w\"ahlen kannst. Lerne hier, welche das sind und wie du darauf zugreifen kannst.
		\input{articles/plot.tex}
		\input{articles/plot3d.tex}
		\input{articles/2dplots.tex}
		\input{articles/3dplots.tex}
		\input{articles/drawing.tex}
		\input{articles/3ddrawing.tex}
		\input{articles/gradient.tex}
		\input{articles/gradient3d.tex}
		\input{articles/vect.tex}
		\input{articles/vect3d.tex}
		\input{articles/compose.tex}
		\input{articles/subplot.tex}
		\input{articles/plotoptions.tex}
		\input{articles/tickformatting.tex}
		\input{articles/color.tex}
		\input{articles/coords.tex}
		\input{articles/linestyles.tex}
	
	\chapter{Datenverarbeitung}
		Ein bedeutender Anteil der Funktionalit\"aten sind solche, die sich auf Datenverarbeitung fokussieren. Dazu geh\"ort Sortieren, Erstellen von Histogrammen, Audiodatenverarbeitung, Fouriertransformationen, Fitten und vieles weiteres.
		\input{articles/sort.tex}
		\input{articles/stats.tex}
		\input{articles/histogramm.tex}
		\input{articles/units.tex}
		\input{articles/audio.tex}
		\input{articles/imread.tex}
		\input{articles/eval.tex}
		\input{articles/datagrid.tex}
		\input{articles/datamodification.tex}
		\input{articles/regularize.tex}
		\input{articles/rotate.tex}
		\input{articles/detect.tex}
		\input{articles/extrema.tex}
		\input{articles/zeroes.tex}
		\input{articles/fft.tex}
		\input{articles/stfa.tex}
		\input{articles/fwt.tex}
		\input{articles/fitting.tex}
		\input{articles/fitting_restrictions.tex}
		\input{articles/fitting_chimap.tex}
		\input{articles/pulse.tex}
		\input{articles/random.tex}
	
	\chapter{Mathematische Operationen}
		Dieses Kapitel dreht sich um umfangreichere mathematische Operationen wie Integrieren, Differenzieren, Matrix-Operationen und Vergleichbares.
		\input{articles/integrate.tex}
		\input{articles/diff.tex}
		\input{articles/taylor.tex}
		\input{articles/spline.tex}
		\input{articles/matop.tex}
		\input{articles/matop_functions.tex}
		\input{articles/pso.tex}
		\input{articles/odesolve.tex}
		
	\chapter{Kontrollfluss}
		Ein Verst\"andnis der Kontrollflussbl\"ocke ist auf dem Weg zur fortgeschrittenen Automatisierung unabdingbar. Lerne hier die sechs zentralen Kontrollflussbl\"ocke kennen.
		\input{articles/flow_ctrl.tex}
		\input{articles/if_cond.tex}
		\input{articles/for_loop.tex}
		\input{articles/while_loop.tex}
		\input{articles/switch.tex}
		\input{articles/try_catch.tex}
		\input{articles/abort.tex}
	
	\chapter{Automatisierung}
		Von einfachen Scripten bis hin zu komplexen Prozeduren findest du hier alles, was dein Herz bez\"uglich Automatisierung begehrt. Lerne hier auch, wie du dateispezifische Konstanten definieren kannst und wie du deine L\"osungen in Packages und Plugins b\"undeln kannst, so dass andere sie verwenden k\"onnen.
		\input{articles/script.tex}
		\input{articles/prompt.tex}
		\input{articles/define.tex}
		\input{articles/declare.tex}
		\input{articles/include.tex}
		\input{articles/procedure.tex}
		\input{articles/procedure_commands.tex}
		\input{articles/install.tex}
		\input{articles/plugins.tex}
		\input{articles/progress.tex}
	
	\chapter{Graphische User Interfaces}
		Wenn du deinen Automatisierungen noch den finalen Feinschliff verpassen willst, bist du bei den graphischen User Interfaces genau richtig. Verwende einfache Dialoge f\"ur schnelle Benutzerinteraktionen und komplexe Window-Layouts f\"ur das komplette Look 'n Feel ausgereifter Applikationen.
		\input{articles/dialog.tex}
		\input{articles/window.tex}
	
	\chapter{Netzwerk und Datenbanken}
		Das Arbeiten mit Remote-Datenquellen ist in der modernen Datenanalyse unabdingbar und wird in diesem Kapitel erl\"autert.
		\input{articles/url.tex}
		\input{articles/mail.tex}
		\input{articles/database.tex}
		
	\chapter{Verschiedenes}
		In diesem Kapitel findest du weitere Kommandos und Funktionalit\"aten, die noch nicht in den anderen Kapiteln behandelt wurden.
		\input{articles/execute.tex}
		\input{articles/additionalcommands.tex}
		\input{articles/language.tex}
		\input{articles/latex.tex}
		\input{articles/documentation.tex}
\end{document}


\end{document}


\end{document}


\end{document}

