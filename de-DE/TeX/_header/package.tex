% package.tex
% --------------------------------------------
% Headerdatei, die alle nötigen Packages lädt.
% ============================================
%
%
%  		Festlegen der Zeichencodierung des Dokuments und des Zeichensatzes.
%     Wir verwenden 'uft8' für das Dokument, da wir seit TxC standardmäßig UTF8-Dokumente erzeugen. Als Legacy-Option kann auch 'Latin1' 
%			(ISO-8859-1) für das Dokument gesetzt werden, wenn die Dateien in ANSI-Codierunge gespeichert werden.
%     Außerdem 'T1' Codierung für die Schrift.
%
\usepackage[utf8]{inputenc}
\usepackage[T1]{fontenc}
%
%  		Paket für die Lokalisierung ins Deutsche laden.
%     Wir verwenden neue deutsche Rechtschreibung mit 'ngerman'.
%
\usepackage[ngerman]{babel}
%
% 		Paket für Farben an verschieden Stellen. 
%     Wir definieren zusätzliche benannte Farben im 'style-header'.
%
\usepackage{color}
%
%			Paket für das Layout der Literaturangabe und 'Deutsche' Literaturangaben.
%
\usepackage[sort&compress, numbers]{natbib}
%
%			Paket für die \subfloat[]{}-Umgebung.
%
\usepackage[margin=1em, indention=1em]{subfig}
%
% 		Das Hyperref-Package verwenden. Spezielle Optionen werden gesondert erläutert.
%
\usepackage[
	pdfauthor={cand. phys. Erik Hänel},%				     											Autor des PDF Dokuments.
	pdfcreator={MiKTeX, LaTeX with hyperref and KOMA-Script},% 								Erzeuger/Kompiler des PDF Dokuments.
	bookmarksopenlevel={sections},%															 								PDF-Lesezeichen bis zu Sections öffnen.
	pdfpagemode={UseOutlines},%                                  								PDF-Lesezeichen beim Öffnen anzeigen.
	pdfdisplaydoctitle={true},%                                  								Dokumenttitel statt Dateiname in der Titelzeile anzeigen.
	pdfstartview={FitH}%																			 								Größe an Fensterbreite anpassen
]{hyperref}
%
%			Das Caption-Paket erlaubt, die Bildunterschriften der \subfloat[]{}-Umgebung zu formatieren.
%
\usepackage{caption}
\usepackage{array}		
\usepackage{graphicx}
\usepackage{tikz}
\usetikzlibrary{snakes}
\usepackage{epsfig}
%
%			AMS-MATH-Package mit der Option, die Grenzen bei Summen, Integralen und anderen Operatoren immer über und unter dem Symbol anzuzeigen.
%
\usepackage[sumlimits,intlimits,namelimits]{amsmath}
%
%			AMS-Packages für spezielle Symbole und Kommutative Diagramme
%
\usepackage{amssymb,amscd}
%
%			Symbole für Körper und Mengen (C,R,N,Q,etc.)
%
\usepackage{dsfont}
%
%			Schriftartensetup: Palatino für Textsatz und Pazomath für Mathesatz
%
\usepackage{mathpazo}
%
%			Schriftartensetup: Helvetica als serifenlose Schriftart und Courier für \texttt{}
%
\usepackage[scaled = 0.95]{helvet}
\usepackage{courier}
%
%			Spezialpackage: Erlaubt es, sich eigene Mathe-Akzente zu definieren. Findet hier bei dem Befehl \vhat{} Anwendung, der ein Hütchen mit 
%			einem Pfeil kombiniert.
%
\usepackage{accents}
%
%			Erweitert die Darstellung von Tabellen mit den Befehlen \toprule, \midrule & \bottomrule
%
\usepackage{booktabs}
%
%			Ermöglicht das direkte Tippen von Kommas ',' als Dezimaltrennzeichen. Erkennt dann aber keine Aufzählungen à la 1, 2, 3, ... mehr.
%
\usepackage{ziffer}
%
%			Dieses Package wird für den Befehl \mum verwendet, der 'µm' als Einheit definiert.
%
\usepackage{textcomp}
\usepackage{epstopdf}
%
% ========================================================
% EOF
%