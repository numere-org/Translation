%	command.tex
% -------------------------------------------
%	Headerdatei, die alle nötigen Befehle lädt.
% ===========================================
%
% QM-spezifische Befehle: \bra{} (\tbra{}), \ket{} (\tket{}), \skal{}{} (\tskal{}{}) [Skalarprodukt], \EW{}, \ew{} [Erwartungswert]
%
\newcommand{\bra}[1]{\left\langle #1 \right|}
\newcommand{\tbra}[1]{\langle #1|}
\newcommand{\ket}[1]{\left| #1\right\rangle}					
\newcommand{\tket}[1]{| #1 \rangle}										
\newcommand{\skal}[2]{\left\langle #1 | #2 \right\rangle}
\newcommand{\tskal}[2]{\langle #1|#2\rangle}					
\newcommand{\EW}[1]{\left\langle #1 \right\rangle}		
\newcommand{\ew}[1]{\langle #1\rangle}
\newcommand{\vhat}[1]{\accentset{\hspace{1.6pt}\text{\fontsize{4.5pt}{4pt}{$\boldsymbol\wedge$}}\hspace{-3.9pt}\text{\fontsize{5pt}{4pt}{$\boldsymbol{\rightarrow}$}}}{#1}\hspace{0.4pt}}
%
% Allg. mathematische Befehle: \msakl [Skalarprodukt: <*,*>], \dx, \e, \eps, \prt{}{}, \Diff{}, \diff{}{}, \svec{}, \abs{}, \norm{}, \rot, 
%	\laplace
%
\newcommand{\mskal}[2]{\left\langle #1 , #2 \right\rangle}
\newcommand{\dx}{\:\text{d}}
\newcommand{\e}[1]{\:\text{e}^{#1}}
\newcommand{\eps}{\varepsilon}
\newcommand{\prt}[2]{\frac{\partial #1}{\partial #2}}
\newcommand{\diff}[2]{\frac{\text{d} #1}{\text{d} #2}} 
\newcommand{\Diff}[1]{\diff{}{#1}}									
\newcommand{\svec}[1]{\begin{pmatrix} #1 \end{pmatrix}} 
\newcommand{\detm}[1]{\begin{vmatrix} #1 \end{vmatrix}}
\newcommand{\abs}[1]{\left|#1\right|}									
\newcommand{\norm}[1]{\left\|#1\right\|}							
\DeclareMathOperator{\rot}{rot}
\newcommand{\laplace}{{}\hspace{-0.3em}\mathrel{\rotatebox[origin=c]{180}{$\nabla$}}\hspace{-0.3em}}
%
% Abkürzungen für Mathematische Körper und Mengen
%
\newcommand{\N}{\mathds N}
\newcommand{\Z}{\mathds Z}
\newcommand{\R}{\mathds R}
\newcommand{\Q}{\mathds Q}
\newcommand{\C}{\mathds C}
\newcommand{\K}{\mathds K}
\newcommand{\V}{\mathds V}
\newcommand{\I}{\Im}
%
% Definition für µm
%
\newcommand{\mum}{\textmu m}
%
% Für Isotopenschreibweise von Elementen/Nukliden z.B. \chem{12}{6}C (Im gewöhnlichen Text)
% \tx
\newcommand{\chem}[2]{\shortstack[r]{\tiny{${#1}$}\\\scriptsize{$_{#2}$}}}
%
% Aufrecht gesetzter Indextext
%
\newcommand{\tx}[1]{_{\mathsf{#1}}}
%
% Der \vldt{} Befehl: Zeichet eine Linie, die das Argument als eine Marginalie auf dem Rand und über dem Strich darstellt. Spezialbefehl zum 
%	Datums-Ordnen von Vorlesungskripten
%
\newcommand{\vldt}[1]{\rule{0mm}{8mm}\textsf{\small{\textcolor{HeadColor}{$\blacktriangledown$ \textbf{Vorlesung vom #1}}}}
\marginline{\fbox{\textsf{\textbf{\small{\textcolor{HeadColor}{#1}}}}}}\newline\rule[.75\baselineskip]{\linewidth}{0.5pt}}
%
%  Abkürzung für den \displaybreak[1]. Mit diesem Befehl kann man 'align'-Umgebungen beim Seitenende gezielt umbrechen.
%
\newcommand{\br}{\displaybreak[1]}
%
% Eine Erweiterung des \ref{label}-Befehls: Generiert einen Link, der Aus Objekttyp, Objektidentifikator und Objektcaption zusammengesetzt ist. 
% Bietet sich für Bezüge auf Abschnitte an: \fullref{sec:abschnitt} produziert z.B. 'Abschnitt 1.1 Abschnittstitel'
%
\newcommand{\fullref}[1]{\autoref{#1} \nameref{#1}}
%
% Umgebung für eine alphabetisch nummerierte Liste
%
\newcounter{ale}
\newcommand{\abc}{\item[\alph{ale})]\stepcounter{ale}}
\newenvironment{liste}{\begin{itemize}}{\end{itemize}}
\newcommand{\aliste}{\begin{liste} \setcounter{ale}{1}}
\newcommand{\zliste}{\end{liste}}
\newenvironment{abcliste}{\aliste}{\zliste}
%	Setzen mit '\aliste\abc Text \zliste'
%
% ============================================
% EOF
%
