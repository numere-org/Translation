% style_phys.tex
% -------------------------------------------------------------------
% Headerdatei, die das Aussehen und das Layout der Dokumente steuert.
% ===================================================================

%
%  1. Definieren von eigenen benannten Farben.
%     Für spätere Verwendung in dem Dokument, definieren wir einzelne
%     benannte Farben. Sie finden bei den Links, den Überschriften und die spezielle
%			'shadecolor' bei den farbigen Boxen (mit '\begin{shaded}\end{shaded}' gesetzt) Verwendung.
%
\definecolor{LinkColor}{rgb}{0,0,0.6}
\definecolor{HeadColor}{rgb}{0,0,0.5}
\definecolor{shadecolor}{rgb}{0.85,0.95,1}

%
%  2. KOMA-Script Option, Zeilenumbruch bei Bildbeschreibungen.
%
\setcapindent{1em}

%
%  3. Stil der Kopf- und Fusszeilen.
%     Wir aktivieren mit 'headings' laufende Seiten- /Kolumnentitel.
%
\pagestyle{headings}

%
%  4. Definieren der Schriftarten/Schiftfamilien der einzelnen Bereiche.
%
\setkomafont{chapter}{\huge\scshape\color{HeadColor}}       				% Kapitel farbig, Groß und als Kapitälchen
\setkomafont{section}{\Large\color{HeadColor}}											% Sections farbig und größer als normal
\setkomafont{captionlabel}{\sffamily\bfseries\color{HeadColor}}     % Fette Beschriftungstitel, farbig und serifenlos 
\setkomafont{caption}{\sffamily}																		% serifenlose Beschriftungen (tables/figures)
\setkomafont{pagehead}{\sffamily\bfseries\color{HeadColor}}         % Kolumnentitel serifenlos, farbig und fett
\setkomafont{descriptionlabel}{\sffamily\bfseries} 									% Fette und serifenlose Beschreibungstitel
\setkomafont{pagenumber}{\sffamily\bfseries}												% Fette, serifenlose Seitenzahlen
\setkomafont{part}{\Huge\scshape\color{HeadColor}}									% 'Part'-Titel Riesig, Kapitälchen und farbig

% Layouteinstellungen für "Subfloats": \subfloat[Titel]{\includegraphics[width=0.5\textwidth]{Pfad/zur/bilddatei.endung}}
\captionsetup[subfigure]{font={sf,footnotesize}, labelformat=simple, labelsep=colon}

%
%  5. Farbeinstellungen für die Links im PDF Dokument.
%
\hypersetup{%
	colorlinks=true,%        Aktivieren von farbigen Links im Dokument (keine Rahmen)
	linkcolor=LinkColor,%    Farbe festlegen.
	citecolor=LinkColor,%    Farbe festlegen.
	filecolor=linkColor,%    Farbe festlegen.
	menucolor=LinkColor,%    Farbe festlegen.
	urlcolor=LinkColor,%     Farbe von URL's im Dokument.
	bookmarksnumbered=true%  Überschriftsnummerierung im PDF Inhalt anzeigen.
}

%%%%%%%%%%%%%%%%%%%%%%%%%%%%%%%%%%%%%%%%%%%%%%%%%%%%%%%%%%%%
%
%  6. Die Besondere Formatierung der Kapitelüberschriften (Linien drüber und drunter)
%
 
% 1st get a new command
\newcommand*{\ORIGchapterheadstartvskip}{}%
% 2nd save the original definition to the new command
\let\ORIGchapterheadstartvskip=\chapterheadstartvskip
% 3rd redefine the command using the saved original command
\renewcommand*{\chapterheadstartvskip}{%
  \ORIGchapterheadstartvskip
  {%
    \setlength{\parskip}{0pt}%
    \noindent\rule[.3\baselineskip]{\linewidth}{1pt}\par
  }%
}
 
% see above
\newcommand*{\ORIGchapterheadendvskip}{}%
\let\ORIGchapterheadendvskip=\chapterheadendvskip
\renewcommand*{\chapterheadendvskip}{%
  {%
    \setlength{\parskip}{0pt}%
    \noindent\rule[.3\baselineskip]{\linewidth}{1pt}\par
  }%
  \ORIGchapterheadendvskip
}
%
%  End of chapter head change
%
%%%%%%%%%%%%%%%%%%%%%%%%%%%%%%%%%%%%%%%%%%%%%%%%%%%%%%%%%%%%
%
% ===========================================================================
% EOF
%